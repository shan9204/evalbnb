% NOTES:
% manually added german quotationmarks: \glqq \grqq // or use UTF-8 characters directly: „quotation marks“
% automatically added quotationmarks: \enquote{Text} // also supports nesting: \enquote{Test with \enquote{nested} quotation.}
% citation: \cite{Pohl2005} -> [1] /// \cite[S. 9ff]{Pohl2005} -> [2, S. 9ff]
%           (ISO-)Standards: http://itooktheredpill.dyndns.org/2010/citing-iso-standards-in-latex-using-bibtex/
% use of "et al.": http://blog.apastyle.org/apastyle/2011/02/et-al-when-and-how.html
%                  First occurence: Number of authors >= 6: use "et al.": "Sunderland et al. explained that.."
%                                   Number of authors <  6: name surnames of all authors
%                  Further occurence: Number of authors >= 3: use "et al.": "Sunderland et al. explained that.."
% references: Add label with \label{label1} and then refer to it with e.g. \hyperref[label1]{Label \ref*{label1} on page \pageref*{label1}}
%                                                                        or conveniently via \myRef{Abschnitt}{label}
% bibliography: keep information minimal (no URLs, no DOI, no ISBN)
% acronyms: \ac{KDE}  == K Desktop Environment (KDE) // on first occurence; on further use only acronym
%           \acs{KDE} == KDE // plural: \acsp{KDE} == KDEs
%           \acf{KDE} == K Desktop Environment (KDE) // \acfp{KDE} == K Desktop Environments (KDEs) 
%           \acl{KDE} == K Desktop Environment // plural: \aclp{KDE}
%       (!) \acp{KDE} == K Desktop Environments (KDEs) (on first occurrence) == KDEs (on further occurrences)
%           Definition: \acro{DD}{Data-Dictionary} // standard acronym definition
%                       \acrodefplural{DD}{Data-Dictionaries} // special plural definition (usually only an 's' is appended: \acp{KDE} == KDEs)
%                       Usage: \acl{DD} == Data-Dictionary // \aclp{DD} == Data-Dictionaries
% non-breaking space: used within abbreviations like z.\,B. to prevent both parts from being separated (e.g. due to line breaks)
%                     \, creates a narrow non-breaking space e.g. used for dates: 23.\,05.\,2042
%                     ~ creates a full non-breaking space e.g. used for full dates (23.\,Mai~2012)
%                        or terms that must not be separated (Prof.\,Dr.\,Bruno~Müller-Clostermann)
% wording: prefer the term "part" or "paragraph" to "chapter" unless you are writing a book or PhD thesis ;-) (German: "Abschnitt" statt "Kapitel" verwenden)
% LaTeX-Tabus: ftp://ftp.dante.de/tex-archive/info/german/l2tabu/l2tabu.pdf 
%
%
% PROBLEMS AND SOLUTIONS:
% Package hyperref Warning: Token not allowed in a PDFDocEncoded string... removing `math shift' on input line XXX.
%    Solution: provide alternative plain text for mathsymbols or special characters that will end up in links. 
%    E.g. change \section{The joy of $\pi$} to \section{The joy of \texorpdfstring{$\pi$}{pi}}
%    http://homepages.inf.ed.ac.uk/imurray2/compnotes/latex.html
% Biliography is not created properly
%    Problem: References are not found after building, log file and \printbibliography command show messages such as "empty bibliography".
%    Try to run biber manually on command line. Possible biber caching error (encountered on OS X 10.8.2):
%    "data source /var/folders/m6/bn7r45zx6cx55rr9g4qh6s6w0000gn/T/par-agoldst/<..>/inc/lib/Biber/LaTeX/recode_data.xml not found in ."
%    Solution: sudo rm -rf /var/folders/m6/bn7r45zx6cx55rr9g4qh6s6w0000gn/T/par-agoldst/
%    see http://humtex.wordpress.com/2011/11/29/biber-first-aid-for-data-source-not-found/
%
%
%
% Config for Mac (OS X 10.8.2 Mountain Lion):
% - Eclipse + TeXlipse + Bibsonomy + MacTeX + EGit
% - Eclipse Word-Wrap plug-in and TeXlipse fix from http://dev.cdhq.de/eclipse/word-wrap/
% - set everything to UTF-8 (Eclipse Preferences > General > Workspace > Default Encoding: UTF-8)
% - Texlipse > Builder Settings > Environment: PATH = /usr/local/bin:/usr/texbin:/usr/bin
% - Texlipse > Builder Settings: set everything to e.g. /usr/local/texlive/2012/bin/x86_64-darwin/
%                                search for ps2pdf and dvipdf via "which ps2pdf": /usr/local/bin/
% - http://www.20papercups.net/latex/latex-texclipse-and-eps-figures/
% - create local Git repository for project
% - synchronise project folder and Git repository with Dropbox (e.g. with a bash script)
% - immediately jump to position in PDF4Eclipse viewer by using Forward-Search: CMD+ALT+F

% ########################### DOCUMENTCLASS ###########################

\documentclass[paper=a4, % equivalent to a4paper
               headsepline, % line after headline
               fontsize=12pt,
               headings=small,
               %draft, % debug mode
               oneside, % one sided document: no alternating page number positions and margins (equivalent to twoside=false)
               ngerman, % new German spelling rules (important for hyphenation)
               captions=tableheading, % within table environment: put \caption{} text above table
               toc=bibliographynumbered, % adds bibliography to table of contents (numbered like chapters; unnumbered with toc=bibliography)
               1.5headlines]
               {scrreprt}

% ######################### PACKAGES + CONFIG #########################

\usepackage[T1]{fontenc} % German umlauts, special chars, etc.
\usepackage[utf8]{inputenc} % input encoding (preferably utf8 with utf8 encoding within your editor! ansinew, latin1, etc. are also possible but require you to use cumbersome workarounds to write German umlauts e.g. "u for writing an ü)
\usepackage[ngerman]{babel} % new German spelling rules (important for hyphenation)
\usepackage[german]{varioref} % for commands like \vref to create references like "...see figures 3.4 to 3.6 on pages 23–24..."


% special symbols
\usepackage{pifont}

% strings have to be entered after utf8 has been set as input encoding

% strings for the frontpage
\newcommand{\documentType}{Masterarbeit} % Art der Arbeit: Master-/Bachelor-/Seminararbeit/etc.
\newcommand{\documentHeading}{Titel}
\newcommand{\documentLocation}{Essen} % also used as location for declaration of academic honesty / Ort in Eidesstattlicher Erklärung
\newcommand{\documentDate}{\today}%10.\,04.\,2014} % also used as date for declaration of academic honesty / Datum in Eidesstattlicher Erklärung

% PDF document metadata (the following strings must not contain line breaks, non-breaking spaces, etc.)
\newcommand{\documentPDFauthor}{Max Mustermann}
\newcommand{\documentPDFtitle}{\documentHeading}



\usepackage{acronym} % enables acronym environment ftp://ftp.tu-chemnitz.de/pub/tex/macros/latex/contrib/acronym/acronym.pdf
                     % http://de.wikibooks.org/wiki/LaTeX-W%C3%B6rterbuch:_Abk%C3%BCrzungsverzeichnis

\usepackage{hyphenat} % hyphenation
\hyphenation{} % special hyphenation for certain words (you can also simply add words that must not be hyphenated)

\usepackage{eurosym} % Euro currency symbol: use \EUR{10} or 10 \euro (\EUR{10} recommended because of correct space between number and symbol)
\DeclareUnicodeCharacter{20AC}{\euro} % UTF-8: enables use of Euro currency sign € (cf. http://www.theiling.de/eurosym.html)

\usepackage{amsmath, amsthm, amssymb} % AMS math symbols and theorem boxes; configure via \newtheoremstyle (cf. ftp://ftp.ams.org/pub/tex/doc/amscls/amsthdoc.pdf)
\usepackage{mathtools}

\usepackage[top=2.5cm, bottom=2cm, left=5cm, right=1.4cm, headsep=1cm, footskip=1cm]{geometry} % borders
%\renewcommand{\restoregeometry}{\newgeometry{top=2.5cm, bottom=2cm, left=5cm, right=1.4cm, headsep=1cm, footskip=1cm}}
    % examples from http://www.artofproblemsolving.com/Wiki/index.php/LaTeX:Layout
    % headsep  = Distance from bottom of header to the body of text on a page.
    % footskip = Distance from bottom of body to the bottom of the footer.
\usepackage{scrpage2} % header/footer http://latex.tugraz.at/latex/tutorial

\usepackage{setspace} % enables line spacing via \singlespacing, \onehalfspacing, \doublespacing or \setstretch{<factor>} for custom spacing
\usepackage{verbatimbox}
\usepackage{rotating} % rotate objects e.g. with \rotatebox{90}{}; also used for tables in landscape format with sidewaystable environment
\usepackage{tabularx}
\usepackage{calc}
\usepackage{array}
\usepackage[font={bf, small, sf}]{caption} % customize the captions in floating environments
\usepackage{subcaption}
\usepackage[final]{pdfpages} % inclusion of external PDF documents (options e.g.: selected pages including a blank one: pages={3,{},8-11,15} )
\usepackage{multirow} % span column over multiple rows
\usepackage{booktabs} % table rules with various thicknesses (\toprule, \midrule, \bottomrule, etc.)
\usepackage{longtable, tabu} % tables that continue to the next page; http://mirror.unl.edu/ctan/macros/latex/contrib/tabu/tabu.pdf 
\usepackage{ltcaption} % caption for longtable
\usepackage{multicol} % add areas with multiple columns
\usepackage{enumitem} % prefix for enumerations
\setlistdepth{9} % enumitem: enable long nested lists
\setlist[itemize,1]{label=$\bullet$}
\setlist[itemize,2]{label=$\bullet$}
\setlist[itemize,3]{label=$\bullet$}
\setlist[itemize,4]{label=$\bullet$}
\setlist[itemize,5]{label=$\bullet$}
\setlist[itemize,6]{label=$\bullet$}
\setlist[itemize,7]{label=$\bullet$}
\setlist[itemize,8]{label=$\bullet$}
\setlist[itemize,9]{label=$\bullet$}
\renewlist{itemize}{itemize}{9}

%\bibliographystyle{alpha} % examples: http://amath.colorado.edu/documentation/LaTeX/reference/faq/bibstyles.html
%\bibliographystyle{dinat} % bibliographies following the german norm DIN 1505
\usepackage[bibstyle=numeric,
            citestyle=numeric,
            backend=biber,
            maxcitenames=2,
            maxbibnames=100,
            sortcites=true,
            doi=false,
            isbn=false,
            url=false]{biblatex} % bibliography package; replaces old natbib package; numeric = Reference in brackets [1].
                                        % remember to translate via "biber file.bcf" (or configure LaTeX editor)
                                        % very useful for printing parts of the bibliography within the document
                                        % see http://www.komascript.de/node/1220 for filter examples
                                        % http://tex.stackexchange.com/questions/32851/biblatex-combine-filters
                                        % http://biblatex.dominik-wassenhoven.de/download/DTK-2_2008-biblatex-Teil1.pdf
%\nocite{*} % display all entries in bibliography regardless of whether they have been cited or not
\DefineBibliographyStrings{ngerman}{andothers={et~al\adddot}} % "u.a." zu "et al."

\usepackage[german=quotes]{csquotes} % dynamic and configurable quotationmarks that \enquote{also support \enquote{nested} quotations.}


\usepackage{lmodern} % Latin Modern fonts
%\usepackage{picins} % insert pictures into paragraphs

\usepackage{float}
\usepackage{wrapfig} % wrap text around figures with wrapfigure environment (has to be loaded right after float package)
\restylefloat{figure} % float only at the place where you define it (without LaTeX moving it)
\floatstyle{plain} % standard: no box is drawn


\usepackage{color} % define own colors: \definecolor{MyColor}{rgb}{0.37,0.17,0.97}
\usepackage{xcolor}
\usepackage{colortbl} % color tables and cells
\usepackage{adjustbox} % http://mirror.unl.edu/ctan/macros/latex/contrib/adjustbox/adjustbox.pdf

\usepackage{framed} % framed or shaded regions that can break across pages

\usepackage{graphicx} % enhanced support for graphics

\usepackage{url} % e.g. \url{http://www.example.com}

\usepackage{fullwidth} % full width environment to display something beyond the text width http://www.ctan.org/tex-archive/macros/latex/contrib/fullwidth


% hyperref package has to be loaded after the float package
\usepackage[%pdftex=true, % only if you use pdftex
            hypertexnames=true,
            pdfstartview={FitH}, % fit the width of the page to the window
            colorlinks=true,
            linkcolor=black,
            citecolor=black,
            filecolor=black,
            urlcolor=black]{hyperref} % clickable urls in PDF: \href{URL}{text}
\hypersetup{% configure PDF metadata with \hypersetup command to pass UTF-8 characters (e.g. umlauts) properly
            pdftitle={\documentPDFtitle}, % title (PDF metadata)
            pdfauthor={\documentPDFauthor}, % author (PDF metadata)
}
\urlstyle{tt} % font style of url: rm = CM Roman; sf = CM Sans Serif; tt = CM Typewriter; same = selected font


% Pseudo Sourcecode Formatting with algorithmicx (http://ctan.org/pkg/algorithmicx)
%   Packages have to be loaded after float and hyperref
%   Alternative: algorithm2e which cannot split code on multiple pages but has nice vertical lines
%   Get vertical lines in algorithmicx as seen in algorithm2e: http://tex.stackexchange.com/questions/110431/ploblems-with-vertical-lines-in-algorithmicx
%   Chaining procedure calls: http://tex.stackexchange.com/questions/16046/how-to-nest-call-in-algorithmicx
\usepackage{algpseudocode} % part of algorithmicx
\usepackage[Algorithmus]{algorithm} % float environment algorithm, parameter is the prefix title in every algorithm box (e.g. "Algorithm 1")


% margin notes
\setlength{\marginparwidth}{1.2in} % width of margin notes
\reversemarginpar % margin notes (added via \marginpar{margin text}) are placed on the opposite side (left/inside instead of right/outside)


% load custom commands and strings
% suppresses page numbering in the header for one page (usage e.g.: \suppressPagenumber{\part{Stand der Forschung}})
\newcommand{\suppressPagenumber}[1]{{\ohead[]{} #1 \ohead[\rightline{\thepage}]{\rightline{\thepage}}}}


% own quotation style for a big single quotation within a block of text
\newcommand{\myQuotation}[2]{\begin{quote}\small{\textit{\enquote{#1}}} -- #2\end{quote}}



% side notes with marginpar
\newcommand{\myNote}[1]{\marginpar{\raggedleft \footnotesize \color{gray} \textit{#1}}}



% cite reference with full title ahead
\newcommand{\myCiteTitle}[1]{\citetitle{#1} (\citeauthor{#1}~\cite{#1})}




% Image/Figure
% Usage: \myImg{caption}{label}{width-factor}{path}
%   Example: \myImg{Beispiel eines Datenflussdiagramms}{fig:dfd}{1}{data-flow-diagram.pdf}
% http://www2.informatik.hu-berlin.de/~piefel/LaTeX-PS/Archive-2004/V08-grafikein.pdf
\newcommand{\myImg}[4]{
    \begin{figure}[h!]
        \centering
        \includegraphics[width=\textwidth * \real{#3}]{./img/#4}
        \caption{\label{#2}#1}
    \end{figure}    
}
\newcommand{\myImgCfg}[5]{
    \begin{figure}[#5]
        \centering
        \includegraphics[width=\textwidth * \real{#3}]{./img/#4}
        \caption{\label{#2}#1}
    \end{figure}    
}

% Shortcut for "..Abbildung 2 zeigt.." via "..\myImgRef{fig:screenshot} zeigt.."
%\newcommand{\myImgRef}[1]{\hyperref[#1]{Abbildung~\ref*{#1}}}
\newcommand{\myImgRef}[1]{\hyperref[#1]{Figure~\ref*{#1}}}



% Shortcut for "..Abschnitt 2 beschreibt.." via "..\myRef{Abschnitt}{chap:myrefchapter}.."
\newcommand{\myRef}[2]{\hyperref[#2]{#1~\ref*{#2}}}


% Shortcut for "..auf Seite 7.." via "..\myPageRef{fig:myfigure}.."
%\newcommand{\myPageRef}[1]{\hyperref[#1]{Seite~\pageref*{#1}}}
\newcommand{\myPageRef}[1]{\hyperref[#1]{Page~\pageref*{#1}}}



% Shortcuts for frequently used strings
\newcommand{\sczb}{z.\,B. } % zum Beispiel; added space because those phrases will never be used at the end of a sentence
\newcommand{\scdh}{d.\,h. } % das heißt
\newcommand{\scua}{u.\,a. } % unter anderem






% Mathematics
\newcommand{\powerset}[1]{\mathcal P~\left({#1}\right)}
%\newcommand{\powerset}[1]{\wp \left({#1}\right)}



% Font
% changes font family ptm = Times, phv = Helvetica,.. http://de.wikibooks.org/wiki/LaTeX-W%C3%B6rterbuch:_fontfamily
%\renewcommand{\familydefault}{phv}



% colored box
% see LaTeX - fcolorbox.png
% http://www.lessjunkmorefunk.de/de/node/10
%\definecolor{Gray}{gray}{0.9}
%\newcommand{\inhalt}[1]{\fcolorbox{black}{Gray}{\parbox{\textwidth}{#1}}}


\addbibresource{./referenzen.bib} % ./ for local folder is very important!


% ############################# DOCUMENT ##############################
\begin{document}
\shorthandoff{"} % deactivates shorthand "a for a umlaut (ä) (necessary if quotation marks are used e.g. in source code)
\setcounter{secnumdepth}{3} % numbering up to three level
%\sloppy % LaTeX will always try to create justified text (this may result in bigger gaps between words)

% ############################# FRONTPAGE #############################
\pagestyle{empty} % remove page number

\includegraphics[width=\textwidth]{./img/logo_unidue_paluno_deutsch.pdf}

\begin{sffamily}

\begin{center}

\vspace*{0.7cm}

\normalsize{Institut für Informatik und Wirtschaftsinformatik (ICB)\\%Institute for Computer Science and Business Information Systems (ICB)
Software Systems Engineering\\
Prof. Dr. Klaus Pohl}

%\vspace{\baselineskip}

\vspace*{2.5cm}

\LARGE{\documentHeading}

\vspace*{1cm}

\large{\documentType}

\vspace*{2.5cm}

\normalsize{submitted to the faculty of economics\\
of the University Duisburg-Essen (Campus Essen) by}

\vspace*{\baselineskip}

\normalsize{
Shanmathuran Sritharan \\
Mülheimerstr. 391 \\
46045 Oberhausen \\
Matrikelnummer: 2279009
}

\vspace*{1.5cm}

\small{\documentLocation, den \documentDate}

\vspace*{1.5cm}

\normalsize{
\begin{tabular*}{\linewidth}{p{0.3\linewidth}p{0.7\linewidth}}
\textbf{Betreuung:}      & Dr.\,Zoltán~Adam~Mann\\
\textbf{Erstgutachter:}  & Prof.\,Dr.\,Klaus~Pohl\\
\textbf{Zweitgutachter:} & Prof.\,Dr.\,Michael~Goedicke\\
\\
\textbf{Studiengang:}    & Angewandte~Informatik -- Systems~Engineering (M.\,Sc.)\\
\textbf{Semester:}       & SS 2017
\end{tabular*}
}

\end{center}

\end{sffamily}
\clearpage % calls \pagestyle{empty}
%\onehalfspacing % set one-and-a-half line spacing after frontpage was included
\setstretch{1.1} % a value of 1.25 refers to one-and-a-half spacing and 1.667 to double-spacing (for 10pt)
                 % (LaTeX calculates with a leading of 1.2 at a point size of 10pt: 1.2 * 1.25 = 1.5 line spacing)

% ############################# HEADLINES ######################
\renewcommand{\headfont}{\small} % small headline font size to fit even long headlines onto the page

\clearscrheadfoot
\ohead[\rightline{\thepage}]{\rightline{\thepage}} % headline with number of page on the right hand side
\automark{chapter}

\pagenumbering{roman} % roman numbering: I, II, III, IV, ...

\pagestyle{scrheadings} % defines whether headlines are printed and how they are styled

% ############################## ABSTRACT #############################

%- Was ist das generelle Thema der Arbeit?
%Beispiel: Kontext ist der Schnittpunkt Variabilitätsmodellierung / SPLE / SA.
%
%- Was ist das ganz konkrete Problem, das in der Arbeit behandelt wird? Was ist die Forschungslücke?
%Beispiel: DFDs haben keine Variabilitätserweiterung. Bisherige Lösungen behandeln das Thema nur abstrakt und können eine Konsistenzprüfung im Sinne des SPLE-Gedankens nicht konkret umsetzen.
%
%- Was ist die generelle Vorgehensweise (Methodik, z.B. Art der Studie: Experiment, Literatursuche, …)?
%Beispiel: SLR zum Aufspüren von Problemen und potentiellen Lösungen mit dem Ziel, diese zu einem eigenen Ansatz zu kombinieren.
%
%- Was ist das zentrale Resultat? Was folgt aus dem Resultat (Wert der Arbeit, z.B. für künftige Studien)?
%Beispiel: Ansatz zur SA im SPLE mit Toolunterstützung. Macht den Weg frei für weiterführende Ansätze in dem Rahmen (evtl. ein knackiges Beispiel nennen, was man tun kann).
%
%Dabei sollte die Wortwahl – im Gegensatz zu den von mir gegebenen Stichpunkten – natürlich durchgehend knackig und äußerst präzise sein. Zu jedem Punkt sollten jeweils ein bis zwei Sätze formuliert werden.
%Insgesamt sollte ein Abstract ca. 100-250 Wörter lang sein (grob eine eine halbe Seite bei dem üblichen Format, niemals mehr als eine Seite).
%
%Du solltest dabei daran denken, dass ein Abstract potentiellen Lesern zur Grobeinordnung Deiner Arbeit dient, z.B., um festzustellen, ob die Arbeit für ihre Forschung relevant sein könnte. Außerdem ist ein Abstract das Aushängeschild der Arbeit; er sollte bei am Problem interessierten Lesern (auch bei Gutachtern!) großes Interesse wecken.
%%
%
%
%
\section*{Abstract}
tbd

%
%
%
\section*{Zusammenfassung}
tbd

% ########################### TOC + ACRONYMS ##########################
\setcounter{chapter}{0}

\setcounter{tocdepth}{2} % table of contents depth: 3 means e.g. 4.3.2.1 will be listed, whereas 2 will lead to 4.3.2. as the deepest entry, etc.
{
\sloppy
\tableofcontents 
}

\listoffigures

\listoftables

\chapter*{List of abbreviations}

\begin{acronym}
    \acro{AOP}{Aspect-Oriented-Programming}
    \acro{BPEL}{Business Process Execution Language}
    \acro{CEP}{Complex-Event-Processing}
    \acro{CP}{Constraint Program}
	\acro{CSP}{Constraint Satisfaction Problem}
    \acro{CVL}{Common Variability Language}
    \acro{DSPL}{Dynamische Software Produktlinie} \acrodefplural{DSPL}{Dynamische Software Produktlinien}
    \acro{ECA}{Event-Condition-Action}
    \acro{EMF}{Eclipse Modelling Framework}
    \acro{FM}{Feature-Model}
    \acro{GQM}{Goal Question Metric}
    \acro{IaaS}{Infrastructure as a Service}
    \acro{MAPE}{Monitor, Analyse, Plan, Execute}
    \acro{OVM}{Orthogonal Variability Model}
    \acro{PaaS}{Platform as a Service}
    \acro{QoS}{Quality of Service}
    \acro{RE}{Requirements Engineering}
    \acro{SaaS}{Software as a Service}
    \acro{SE}{Software Engineering}
    \acro{SLA}{Service Level Agreement}
    \acro{SOA}{Service-Oriented Architecture}
    \acro{SPLE}{Software Product Line Engineering}
    \acro{UML}{Unified Modeling Language}
	\acro{VM}{Virtuelle Maschine} \acrodefplural{VM}{Virtuelle Maschinen}
\end{acronym}
\clearpage % Clear page to get the arabic page counting on the next page

% ########################## LAYOUT + DESIGN ##########################

\clearscrheadfoot % cleans footer from previous configurations
\ihead[]{\leftline{\leftmark}} % current section and chapter heading in header on the left
\ohead[\rightline{\thepage}]{\rightline{\thepage}} % page number in header on the right
\automark{chapter} % for creating the current section/chapter in the heading

\pagenumbering{arabic} % arabic numbering: 1, 2, 3, 4, ...
\setcounter{page}{1}

\addtokomafont{chapter}{\LARGE}
\addtokomafont{section}{\Large}
\addtokomafont{subsection}{\large}
\addtokomafont{paragraph}{\itshape}

\renewcommand{\thepart}{}%\Alph{part}}
\renewcommand{\thesubsubsection}{\Roman{subsubsection}}

\renewcaptionname{ngerman}{\partname}{}

% ############################## CONTENT ##############################

%
%
%
\chapter{Introduction}
\label{chap:intro}

%
%
%
\section{Motivation}
tbd

%
%
%
\section{Goals of thesis}
tbd

Zur Erreichung des Hauptziels der Arbeit werden folgende Teilziele definiert:

\begin{itemize}
   \item \textbf{Teilziel 1:} tbd  
   \item \textbf{Teilziel 2:} tbd  
   \item \textbf{Teilziel 3:} tbd  
   \item \textbf{Teilziel 4:} tbd  
\end{itemize}


%
%
%
\section{Vorgehensweise}
\subsection*{Adressierung von Teilziel 1}
tbd

\subsection*{Adressierung von Teilziel 2}
tbd

\subsection*{Adressierung von Teilziel 3}
tbd

\subsection*{Adressierung von Teilziel 4}
tbd
\chapter{Test X}
Test der Abkürzungen.

\ac{DSPL}

\acs{BPEL}

\acl{MAPE}

%
%
%
\section{Test Figure}

Eine beispielhafte Vektorgrafik ist in \myImgRef{fig:test} zu sehen.

\myImg{Test Figure}{fig:test}{0.8}{overview_f.pdf}

Für XYZ siehe \myRef{Tabelle}{tbl:test}.

% lange Tabelle über mehrere Seiten: longtabu
{
\tabulinesep=\tabcolsep
\begin{longtabu} to \textwidth {|X[3]|X[7]|}
\caption{\label{tbl:test}Test 1} \\
\hline
  \textbf{XYZ} & \textbf{ABC} \\
  \hline
  ABC & DEF \\
  \hline
\end{longtabu}
}

\section{Test Referenz}
Beispiel siehe \citeauthor{marquezan_towards_2014} \cite{marquezan_towards_2014}.
% ...more inputs
%\input{99-abc}
% ...
\chapter{Zusammenfassung und Ausblick}

%
%
%
\section{Zusammenfassung}
tbd

%
%
%
\section{Ausblick}
tbd

% ########################### BIBLIOGRAPHY ############################

{
\sloppy
\printbibliography[title={\label{chap:references}Literaturverzeichnis}]{} 
}

% ############################# APPENDIX ##############################

\part*{Anhang}
\newcounter{chapterAppendix}
\setcounter{chapterAppendix}{1}
\renewcommand{\thechapter}{\Alph{chapterAppendix}}
\chapter{Anhang}
\label{chap:appendix}

\section{tbd}
tbd

% ############################ DECLARATION ############################

% declaration of academic honesty / Eidesstattliche Erklärung
\chapter*{Eidesstattliche Erklärung}
\label{chap:declaration}

Hiermit versichere ich, dass ich die vorliegende Arbeit ohne Hilfe Dritter und nur mit den angegebenen Quellen und Hilfsmitteln angefertigt habe. Ich habe alle Stellen, die ich aus den Quellen wörtlich oder inhaltlich entnommen habe, als solche kenntlich gemacht. Diese Arbeit hat in gleicher oder ähnlicher Form noch keiner Prüfungsbehörde vorgelegen.
\vspace*{1cm}
\begin{flushright}\small \underline{\documentLocation, den \documentDate \hspace{0.4\textwidth}}\end{flushright}

\clearpage

% #####################################################################

\end{document}